This chapter will give give an overall outline of the project as a whole, including summaries of the project's idea, its design and implementation, the issues that arose along the way, and the project's outcome in completion.

\section{Project Idea and Management}

    As a project, the development of an architectural visualisation tool for use with virtual reality systems has been not only challenging but also rewarding. The project's success is discussed in detail within chapter~\ref{chp:evaluation} of this report and a reflection upon its scheduling the project's development period is made.
    
    Upon original project inception, the group comprised of four people. Each member got to know each other over the summer period where project ideas were proposed and discussed, eventually the project idea of creating an arch. vis. \acrshort{vr} tool was decided upon and met with excitement to work with a novel technology. A positive working environment was established and study into the feasibility of the project swiftly began with the creation of a prototype. Soon thereafter the fourth member of the group departed the project, causing the group to face many challenges going forward. The groups specification had to be re-worked to better suit the remaining member's technical capabilities, whilst tasks and team roles had to be re-distributed to balance out work load. With the redistribution of tasks being equal, all project members had to also spend time researching best practices for the Unity \acrshort{sdk} and C\# standard library. Once these challenges were overcome, the project ran smoothly with development maintaining pace with the specified project timetable and departmental deliverables. The use of \acrshort{vr} in an industrial setting is a reasonably new subject area, and with \acrshort{vr} systems only coming to market over the past 18 months, considerations had to be made with regards to obtaining access to a system that would meet the project's needs. Two university societies kindly offered their HTC Vive systems with excitement to see what is possible with the hardware beyond entertainment such as video games. Overall the project idea was met with excitement from all involved, and each project member expressed great enthusiasm to with with a new technology that is just reaching mainstream.

\section{Architectural Visualisation Tool}

    The architectural visualisation tool was originally intended not only for use in the creative process of designing a building or space, but also would include a physics simulation that would demonstrate the structural soundness of any structure that is created with the tool. Upon departure of the group's fourth member, the tool was re-purposed to act simply as a means to augment or possible replace the current work pipelines for arch. vis. The software was developed using an Model-View-Controller pattern using the Unity \acrshort{sdk} and C\# language using libraries such as Valve's SteamVR plugin to integrate with the HTC Vive system. A number of tools were devised and an extensible tool framework established to facilitate their development.
    
    There were a few issues when it came to the development of the tool, more specifically surrounding the implementation of the algorithms used to generate walls, doors, and roofs. Whilst these algorithms are widely documented, their implementation in C\# was not available in many libraries, if at all. As a result of this, other well-tested implementations were translated from languages such as JavaScript into C\# for use with Unity. Despite the difficulties and technical challenges faced by the group throughout the development of the project, the group was able to successfully create an architectural visualisation tool for use with a virtual reality system.
    
\section{Reflections}

    Overall, the project was completed successfully with all requirements satisfied and objectives met. This was due to the deliberate and concise planning of the project's tasks, the group member's enthusiasm to work with novel technologies, their resolve to complete assigned tasks within the allotted time, the tenacity of the group as a whole to overcome the numerous challenges introduced with the departure of a key group member, and the robust management of the project in terms of its scheduling and methodology. The project itself can be said to serve as an example usage of virtual reality outside of the entertainment industry. It also can be used as an example project for students who are interested in what can be done to use \acrshort{vr} as a basis for their third and fourth year projects. 

\section{Closing Remarks}

    In final conclusion, the project has been a fantastic learning experience for all individual group members. It has helped each of them develop skills with relation to the planning, design, development, and management of a large team project over an extended period of time. The for the project members to deal with unforeseen problems has grown as well. The support and advice provided by the project supervisors and amongst the group's individual members was fundamental for reaching the level of success that was enjoyed by the project. 