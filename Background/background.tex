This section will introduce the core components for the project into which research will be conducted. It aims to provide the reader with definitions for key words that shall appear throughout the project. These definitions will then be used to help construct a definition of the project's problem domain and its objectives.

\section{Virtual Reality}

    \subsection{Definition}
    
        \acrfull{vr} is the concept wherein a user is presented with a realistic and immersive simulation of a three-dimensional environment that is experienced or controlled with the movement of the body~\cite{Dictionary2017}. \acrshort{vr} systems have been around since the early 2000s with designs such as the \textit{SegaVR} \acrfull{hmd}~\cite{SegaVR2004}. More modern systems such as the HTC Vive or Google's Daydream have since come to market.
    
    \subsection{System Designs and Capabilities}
    
        Modern \acrshort{vr} systems may take one of many different design philosophies when it comes to hardware implementation. HTC's Vive \acrshort{hmd} and controllers use a number of infra-red laser strobing base stations to track a user's location and orientation within the physical world~\cite{LighthouseAnalysis2016}, whilst also requiring a tether to a computer running the appropriate SteamVR software. Google's Daydream uses on-phone sensors to perform the same functionality using a mobile phone without a tether. With the risk of cyber sickness when systems do not respond accurately or fast enough to user input~\cite{LaViola:2000:DCV:333329.333344}, these systems also deliver high tracking accuracy with their implementations. For the purposes of this project, desktop-oriented \acrshort{vr} systems like the HTC Vive will be considered.
    
    \subsection{Usages}
    
        With \acrshort{vr} becoming one of the salient technologies marketed toward gamers, the associated systems are becoming even more affordable and popular. Due to the high precision necessary to avoid inducing cyber sickness in users, the possible usages of these systems extends beyond just computer games. \acrshort{vr} has been used to treat Phantom Limb Pain within people who have lost limbs~\cite{ortiz2014treatment}, or can be used to help train those aiming to become skilled tradesmen~\cite{VrWelding2014}. Whilst research into the uses of \acrshort{vr} as a commercial or industrial tool is in its infancy, it is the wide domain that is already demonstrated shows the strengths of \acrshort{vr} and its potential importance in the future as the hardware and software systems mature.

\section{Human-computer Interaction}

    \subsection{Definition}
    
        \acrfull{hci} is the study of how people interact with computers and to what extent computers are or are not developed for successful interaction with human beings~\cite{HciDefinition2017}, with the aim of producing a system that functional and usable system that is safe for the user. Each interaction will involve the user of the system, the computer that is running the system, and the interaction between both computer and user. Usability within this concerns the ease of use of a particular system as well as its efficiency, effectiveness, and ease of remembering how to use it as well. A usable system is also enjoyable to use.
    
    \subsection{User Interface and Experience}
    
        Within any system that requires a user's input, the user interface is at the intersection of the user and any interaction. To create a usable interface it is crucial to make it efficient and easy to use. On top of this, the experience a user goes through using the interface must be designed to be enjoyable and intuitive. Given an environment that is manipulated in \acrshort{vr}, these qualities are even more important. If a user interface is not efficient it may cause the user to become tired over time. Similarly if an interface is confusing then a user may not want to use the software at all.

\section{Architecture Visualisation}

    \subsection{Definition}
    
        Architecture visualisation is an area within architecture design in which a photo-realistic rendition of a building's interior or exterior is produced from architectural plans. These plans are used to demonstrate to clientele or stakeholders in a project what a project may look like when finished. It is a creative process that also may feed back into the plans for a structure or interior based upon client feedback.
    
    \subsection{Current Workload Pipelines}
    
        The typical workload for rendering or visualising an architectural project will involve creating a 3D model of the given structure from architectural drawings as well as associated assets like chairs and tables. These models are created and scenes are composed using 3D software packages like Autodesk's AutoCAD over a number of weeks. These scenes are then rendered over a period of hours or days and the renders presented to the clients of a project. Feedback is re-integrated into the designs and scenes are modified and rendered, creating a process that may take a few weeks to update a client on their suggestions.