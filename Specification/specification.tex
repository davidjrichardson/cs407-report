% TODO: Section preamble
This section introduces the problem space that this project aims to solve using \acrshort{vr} systems that are available to to consumers. From this problem description a number of objectives for the project will be clearly defined as to form a foundation for the rest of the project. A justification of the proposed solution as to why the project is necessary and valid will then be given. Analysis into the possible stakeholders of the project will then be undertaken. A feasibility study of the project will then be conducted, discussing the scope of both the problem domain and the project itself. The study will also be used as a tool to gain insight with possible problems that may arise throughout. The possible financial impacts of this the proposed solution shall be briefly considered and an introduction to the project's management will also be given. Through this feasibility study it is possible to demonstrate that the project can be completed within the given time frame and with available resources. An analysis of the objectives to create a set of concrete functional and non-functional requirements will then be performed, with the intent that these requirements form a mechanism to measure project progress with each deliverable. These deliverables will then be outlined. Finally any alterations from the original specification document submitted in October 2016 will be identified and justifications for the changes given.


\section{Problem Description}

    Within the context of this project, the problem that it aims to address is the possible use of \acrshort{vr} to augment the creative process for which architectural visualisations are created. As previously outlined, the current methods can take weeks from first concept to a finalised design that has client approval. The introduction of a \acrshort{vr}-based tool would allow for a faster iterative cycle by allowing the client to get feedback to their suggestions near-instantaneously. Such a tool will also add other qualities to the visualisations that are not possible with 2-dimensional images alone. For this project the group will focus on the creation of a prototypical tool for the creation of indoor and outdoor spaces in a virtual reality. To this end, the definition of a problem that can be used to demonstrate the software produced would ideally be the creation of an in-progress building like the new Mathematical Sciences building, or the reconstruction of a pre-existing space such as a room in the Department of Computer Science. A more concrete definition of a scenario that could be used for a demonstration can be given later.
    
    With the group's focus being on the creation of an \acrshort{vr}-based architecture visualisation tool prototype, the decision was made to develop it using the Unity game development tools and the C\# programming language. Within section~\ref{TODO} of the literature review, pre-existing tools that solve a number of this project's requirements are discussed. Overall, it is found that no single solution can be tailored to the overall requirements of the project and as such the decision was made to shift focus towards development of a bespoke tool. This solution can then be split into a number of principle components: The user interface, the tools system, the data representation, and geometry construction algorithms. For a solution to the posed problem, the software must have a means through which a user can interact with it. The tools system will facilitate the ability for a user to craft a structure or space to their liking in \acrshort{vr}. With these in place, the underlying data representation and geometry construction algorithms will provide a means to generating 3D meshes for the virtual world.

\section{Objectives}

    From the problem statement, the overarching aim of this project is the design and development of a prototypical architectural visualisation tool for use with \acrshort{vr} hardware systems like the HTC Vive. The completion of such a tool will grant the ability to demonstrate how the concept of virtual reality can be used to augment the architectural design process, as well as the use of \acrshort{vr} in creative processes as a whole. Early on in the project its core components were defined as part of the specification, and can be used to derive a set of project objectives that are summarised as follows:
    
    \begin{enumerate}
        \item{\textbf{Implement a \acrshort{vr}-based architectural visualisation software.}} Through the use of off-the-shelf \acrfullpl{sdk} designed to create games, or by developing our own rendering and interaction engine specifically tailored to our problem domain.
        \item{\textbf{Allow the user to navigate the virtual world through walking or teleportation.}} The user must be able to move around the world such that they can manipulate or experience it at any point.
        \item{\textbf{Enable the creation and modification of walls, floors, windows, doorways, and roofs on multiple floors.}} To construct any outdoor or indoor spaces in a virtual world, the necessary tools do build basic structures are needed.
        \item{\textbf{Provide mechanisms for saving, restoring, and distributing scenes.}} 
        \item{\textbf{Implement the placement, movement, and removal of static items such as furniture.}}
        \item{\textbf{Supply a number of standard assets that can be placed and manipulate.}}
    \end{enumerate}

    % TODO: List the abstract objectives of the problem
    % TODO: Link to the project management session where appropriate

\section{Justification}

    % TODO: Discuss the justification of the project with respect to its usefulness within the world of VR, Industry
    
\section{Stakeholder Analysis}

    % TODO: Identification of possible stakeholders to the project are (include project supervisors?)

\section{Feasibility Study}

    % TODO: Discussion of how feasible the project is. Should analyse the problem scope, project scope, and analyse the finance and market. The management of the project should also be discussed
    
    \subsection{Problem Scope}
    
    \subsection{Project Scope}
    
    \subsection{Financial Analysis}
    
    \subsection{Market Analysis}
    
    \subsection{Project Management}
    
\section{Requirements Analysis}

    \subsection{Functional Requirements}
    
        % TODO: Create a table for these
    
    \subsection{Non-functional Requirements}
    
        % TODO: Create a table for these

\section{Project Deliverables}
    
    % TODO: Explain all previous deliverables, what's happening in the future (demo) and the stuff that's being handed in with this report

\section{Alterations from the Original Specification}

    % TODO: A large part of the specification was re-worked, this is explained in the project management section