Understanding Virtual Reality by Sherman, et al.~\cite{sherman2003} explain in detail the concept of~\acrfull{vr}. Despite it being quite old, coming out in 2002 the concepts described in it are quite time universal and the book shows it's forwards thinking in describing potential possibilities of using~\acrshort{vr} that correlate closely to today's capabilities or even go beyond them. It covers ideas that are not yet available to today's everyday user, like rendering the user's avatar in~\acrshort{vr} and replicating it's mouth movement, or tracking eyes to allocate more resources to the rendering in more detail the are the user is looking at. The book goes over multiple ways of tracking positions of the body as well as monitoring physiological attributes, like the heart rate o respiration rate.

Even though those are not necessary relevant to our project, the underlying concepts are still applicable to some of the aspects we are working with. It talks for example about user's experience of navigating through a virtual world. It also covers the safety issues that we are still facing today: the dangers of wearing a head-based display like tripping hazard, eye fatigue or nausea. It is also interesting and slightly amusing to read that back in those days a head-based display with cathode ray tube was one of the possible options of implementing a head over display. In the modern day, the idea of wearing CTR displays in front of our eyes feels somewhat surreal.

The book also talks about the lag between the head movement and the scene update. This is directly linked to one of our functional requirements F5 listed in section~\ref{sec:requirements}, which talks about having a high frame rate that would reduce the chance of any negative effects occurring. The book described that those can include dizziness, blurred vision or vertigo, but are in general only short-term and shouldn't have long-lasting effects. It is interesting to note that according to the book, humans are able to acclimate to the lag in the system and start using the system as if the lag wasn't present. This however causes problems when the~\acrshort{vr} head-set is removed and the user is still expecting the lag to be present, which leads to impairment of temporal and spacial judgement for some time, while the brain again gets used to the lag of lag. In the context of our project this may mean that a slight drop below the optimal frame rate would not make the system unusable, but instead make it harder to get used to it. Despite that, the requirement to maintain high frame rate was still considered to be in place.

The rest of the book talks about the different ways of implementing~\acrshort{vr} environment, covering different methods of tracking the user and displaying the world back to them. This was not really relevant to out project as we already had a working implementation in form of HTC Vive to use.

%  http://www.jmir.org/2014/12/e283/?utm_source=feedburner&utm_medium=feed&utm_campaign=Feed%3A+JMedInternetRes+(Journal+of+Medical+Internet+Research+(atom))
One of the papers that talks much more about the use of~\acrshort{vr} in a similar way that we're trying to implement is the paper by Atwal, et al.~\cite{atwal2014}. It even introduces the term~\acrfull{vrida} that described exactly the kind of application we are tying to implement. This particular paper talks about using this sort of tool to prototype changes to a patent's home to verify their usefulness before implementing them. The research quoted by the paper says that ``more than 50\% of specialist equipment installed as part of home adaptations is not used by patients''. Using a~\acrshort{vr} environment allows occupational therapists to show the changes to the patients and get their feedback using a think-aloud protocol to alter the modifications and make them more cost-effective.

The findings show that this approach would increase the collaboration between the practitioner and the patient. The system was proved to be successfully usable in multiple scenarios and allowed the patients to complete the given tasks. It also talks about the restrictions of such an approach. Some patients with cognitively impairments may struggle to use the system, also a number of patients noted that some of the equipment was missing from the library that was used.

In the context of our application, it seems that it might be used for the application above. The system supports building arbitrary rooms, which can represent the patient's house. The object tools also allow the practitioners to add, move and remove objects from the surrounding. The way objects are included in the system, allow for relative easy addition of new specialist equipment to the object library, which directly addresses the complains of some of the patients. Unfortunately, we cannot overcome the accessibility issue, as it is directly linked to~\acrshort{vr} and is beyond out control.

%  https://hal.archives-ouvertes.fr/hal-00919933/file/JVRC2013_Hal_version.pdf
Another paper that talks about feasibility and uses of~\acrshort{vr} for design purposes is a paper by Chellali, et al.~\cite{chellali2013}. It talks about the design process that the designers currently go through and describes their proposed solution to use~\acrshort{vr} to help them with that. Similarly to the previous paper, it introduces a new term for this:~\acrfull{vr4d}. In contrast to our idea however, the suggested approach used a 2D system, with a tablet to create initial design, and then use~\acrshort{vr} to visualise and edit the initial prototype.

The paper states that one of the advantages of their approach is reducing the time and cost by speeding up iterations of the iterative design the designers follow. This lines up with our idea of creating a quick prototyping tool. The paper also talks about the ability for a better real time collaboration between the designers, which is something that we did not implement, but consider an extension of our project. The suggested system and out system share the core functionality of modifying the objects in the scene.

The paper concludes that their approach had positive impact on the collaboration of the designer during the design process. The usefulness of the immersive environment was rated of about 5.2 out of 7 on the scale the researchers used. The paper concentrates more on commenting the tablet part of the system rather than the~\acrshort{vr} aspect of it. The paper states that the author plan to perform further study on the~\acrshort{vr4d} in the next paper, but we were unable to find such paper.