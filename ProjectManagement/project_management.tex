When undertaking any project, several challenges may come to light and subsequently must be rigorously accounted for with appropriate documentation during the planning, design, and implementation phases of the project. Some of these challenges could be the organisation of the team members, the selection of what tools that team members shall use, or the software development methodology itself. The purpose of this chapter is to discuss areas of project management such as these, and how the group and its members overcame challenges like them throughout the project's life cycle.

\section{Project Methodology}

    \subsection{Project Summary}        
    
        To gain a clear understanding of any software development project, a few key tasks must be first complete before any development work can be done. The project aims and resources to achieve them must be clearly defined, and the stakeholders in the project should be identified. Once these steps have been taken, it's also crucial to justify the project to each group member as to ensure that they are individually happy with why and how the project will move forward. For this project this has already been discussed in detail within chapter~\ref{chp:specification}, though this section will contain a summary of this specification from the perspective of project management.
        
        The original project idea was proposed and accepted with intrigue and excitement from all group members after an initial enquiry into its feasibility, given that the hardware to test virtual reality software is not common place. Once this concern was addressed with the Computing Society at the university giving the group access to the HTC Vive \acrshort{vr} system. The project gave the group members an opportunity both to work with a novel \acrshort{hci} technology, and to work on the cutting edge of the research into possible industrial uses of \acrshort{vr}. Using the HTC Vive, the group set out with the objective of creating a tool that can be used to aid the architectural design and visualisation process, allowing the user to create and manipulate 3 dimensional spaces in a virtual reality.
        
        With a large project scope, it was decided to develop the software as a proof-of-concept with a minimal list of tools that provided all of the functionality necessary to meet the project objectives and aims. With the project's approval by its supervisors, the following work consisted of determining a use case to work toward, researching \acrshortpl{sdk} and best practises for developing a virtual reality application, and finalising the list of basic tools to be implemented in the project software. Once these tasks were completed, a list of common household items to be included as assets was drawn up. As described in section~\ref{sec:progress_management}, the progress of the project's stages was monitored, with the tasks being scheduled between members both evenly and to suit each member's individual strengths where possible. The project was tested frequently to also monitor potential software problems and functionality regressions as time went on. In the final stages of the project, the departmental project deliverables were consolidated and brought to a conclusion to ensure that it is delivered safely whilst also addressing the project's previously established objectives.

    \subsection{Software Development Methodology}
    
        As with any software development project, it is necessary to consider enacting an appropriate development methodology. If not, the lack of effective practices can lead to unpredictable functionality, poor code quality and documentation, and repeated error. Software development methodologies range from the rigid-structure, document centric waterfall approach to more agile methods that prioritise adaptability to requirement changes and code quality. With a small project group size of just 3 people who have fluctuating schedules week-upon-week and thus time that can be dedicated to project work, a variation of the scrum agile methodology was chosen. Core concepts of an agile methodology were retained like the frequent delivery of working software, welcoming continuous changes in requirements, and daily cooperation between developers~\cite{martin2003agile}. Other methods such like waterfall were considered inappropriate within the context of this project due to their inflexibility with changing requirements, or for being geared toward more enterprise-like group sizes where teams can be 10s to 100s of developers in size.
        
        With the flexibility of development introduced by the an agile methodology, internal milestones for software development were put in place with the flexibility to cater for changes in requirements. The group structure also reduced the focus on sprint-based development that lies within traditional scrum-based methods due to there being no emphasis on delivering functional products to the project stakeholders on a regular basis. The group structure consists of one development team of 3 individuals, opting to replace frequent (typically daily) scrum periods with sporadic development sprints that are encapsulated by discussion between development team members.
        
        With some parts of the project being independent from others, it was possible to use the scrum methodology effectively to allow two members to complete these tasks simultaneously. Because of this, any iterations upon these independent tasks was also possible without as much planning as would be necessary if they were dependent, helping the fact that the project has a small time frame. Frequent meetings with project supervisors generated feedback that could be acted upon within a short time frame thanks to the agile method that was adopted. It is also important to recognise that the agile method does carry the danger that sufficient documentation of a code base will not be created, thanks to its code-centric approach to development. This affects teams that have members that are constantly in flux. Fixed teams that do not change are unlikely to run into this problem, though it should be made imperative that documentation is monitored to avoid the documentation becoming stale.
        
\section{Team Structure}

    As mentioned in previous sections of this report, the project group consists of 3 members, originally starting with 4. With such a small group of people, it was imperative that the work load was balanced as equally as possible. With the loss of a group member at an early stage in the project, it was also necessary that task distribution was also done intelligently to account for the loss in expertise. Work was also divided in a way to play to each member's individual strengths. With 3 group members it was impractical to try split work between at least two people, but a focus on clear and concise documentation meant that if a group member had to fill in for another, they could swiftly get up to speed with the work. Whilst there are obvious benefits to having a larger number of people in a group, there are also benefits to small groups; organisation is simpler, the allocation of tasks is straightforward, and each group member is more aware of the work they have been assigned. A smaller group does leave a disadvantage when it comes to the sheer number of developers available to work on the project at any one moment. The individual members of this project and their assigned roles are shown below:
    
    \begin{itemize}
        \item \textbf{Alex Dixon - Mesh generation, UI, and software back end architect.} Responsible for the primary mesh generation algorithms for walls, floors and roofs, the user interface, and the abstract representation that is used to generate the meshes.
        \item \textbf{David Richardson - Project manager and environment manipulation.} Responsible for the distribution of tasks between group members and managing deadlines, as well as the implementation of environment lighting manipulation functionality.
        \item \textbf{Jakub Zawadzki - Tools developer and 3D asset builder.} Responsible for the architecture of the tool subsystems and the construction and exporting of 3D assets for inclusion with the final software deliverable.
    \end{itemize}
    
    Whilst this list is not exhaustive, it serves as a good demonstration of the group's ability to allocate tasks to each member and for them to take responsibility for them.

\section{Time \& Task Management}
\label{sec:timetable}

    The effective management of the time allocated to a project is necessary to ensure it is completed within the given time constraints. Because of this it is necessary that tasks are identified and scheduled for completion in an appropriate time frame. Figure~\ref{fig:project_timetable} gives a Gantt chart that gives an outline of the task break down of the project, the scheduling of those tasks, any inter-task dependencies, and important internal milestones for the group throughout the lifecycle of the project.
    
    \begin{sidewaysfigure}
		\centering
		\begin{ganttchart}[
			x unit=5.25mm,
			hgrid,
			vgrid,
			newline shortcut=true,
			time slot format=simple,
			bar/.append style={fill=Aquamarine!25},
			bar label node/.append style={align=right},
			milestone inline label node/.append style={right=3mm},
			milestone/.append style={fill=RubineRed!50},
			today=30,
			today rule/.style= {red, ultra thick, dashed},
			today label={Product delivery},
			link bulge=.5,
			link/.append style={thick}
		]{1}{31}
			\gantttitle{Term 1}{10}
			\gantttitle{Christmas}{4}
			\gantttitle{Term 2}{10}
			\gantttitle{Easter}{5}
			\gantttitle{Term 3}{2} \\
			\gantttitle{Time (in weeks)}{31} \\
			\gantttitlelist{1,...,31}{1} \\
			\ganttbar[name=feasibility]{Feasibility study}{1}{2} \ganttmilestone[inline=true]{First prototype}{7} \ganttmilestone[inline=true]{Final prototype}{20} \ganttnewline
			\ganttbar[name=concept]{Concept design}{3}{4} \ganttmilestone[inline=true]{Second prototype}{10} \ganttmilestone[inline=true]{Feature freeze}{25} \ganttnewline
			\ganttbar[name=uidesign]{UI development}{6}{8} \ganttbar[name=uidesign2]{}{11}{13} \ganttbar[name=uidesign3]{}{23}{24} \ganttnewline
			\ganttbar[name=qa]{Quality assurance}{9}{10} \ganttbar[name=qa2]{}{14}{15} \ganttbar[name=qa3]{}{20}{22} \ganttbar[name=qa4]{}{27}{29} \ganttnewline
			\ganttbar[name=sysdev]{Systems development}{6}{9} \ganttbar[name=sysdev2]{}{12}{15} \ganttbar[name=sysdev3]{}{18}{19}
			\ganttnewline
			\ganttbar[name=assets]{UI development}{16}{17}
			\ganttnewline
			\ganttbar[name=bugfixes]{Bug fixing}{25}{29}
			\ganttlink{feasibility}{concept}
			\ganttlink{concept}{sysdev}
			\ganttlink{concept}{uidesign}
			\ganttlink{sysdev}{sysdev2}
			\ganttlink{uidesign}{qa}
			\ganttlink{qa}{uidesign2}
			\ganttlink{uidesign2}{qa2}
			\ganttlink{qa}{sysdev2}
			\ganttlink{qa2}{sysdev3}
			\ganttlink{qa3}{uidesign3}
			\ganttlink{uidesign3}{bugfixes}
			\ganttlink{sysdev2}{sysdev3}
			\ganttlink{sysdev3}{bugfixes}
			\ganttlink{sysdev3}{qa3}
		\end{ganttchart}
		\caption{Project timetable from week 1 term 1 to week 2 term 3}
		\label{fig:project_timetable}		
	\end{sidewaysfigure}
	
	Along with the development of the software for the project, its deliverables also require scheduling to make certain that they are delivered on time without error. Figure~\ref{fig:document_timetable} shows the timetable breakdown of work toward these deliverables up to the point of hand in for this report.
		
	\begin{sidewaysfigure}
	    \centering
	    \begin{ganttchart}[
			x unit=5.25mm,
			hgrid,
			vgrid,
			newline shortcut=true,
			time slot format=simple,
			bar/.append style={fill=Aquamarine!25},
			bar label node/.append style={align=right},
			milestone inline label node/.append style={right=3mm},
			milestone/.append style={fill=RubineRed!50},
			today=30,
			today rule/.style= {red, ultra thick, dashed},
			today label={Product delivery},
			link bulge=.5,
			link/.append style={thick}
		]{1}{31}
	        \gantttitle{Term 1}{10}
			\gantttitle{Christmas}{4}
			\gantttitle{Term 2}{10}
			\gantttitle{Easter}{5}
			\gantttitle{Term 3}{2} \\
			\gantttitle{Time (in weeks)}{31} \\
			\gantttitlelist{1,...,31}{1} \\
			\ganttbar[name=spec]{Specification}{2}{3} \ganttmilestone[inline=true]{Specification hand in}{4} \ganttnewline
			\ganttbar[name=progress]{Progress presentation}{8}{9} \ganttmilestone[inline=true]{Progress presentation hand in}{10} \ganttnewline
			\ganttbar[name=finalreport]{Final report}{23}{30} 
	    \end{ganttchart}
	    \caption{Documentation timetable from week 1 term 1 to week 2 term 3}
	    \label{fig:document_timetable}
	\end{sidewaysfigure}
		
\section{Progress Management}
\label{sec:progress_management}

    \subsection{Meetings}
    
        Regular meetings were scheduled between the project group and supervisors every fortnight. These meetings were used to update the project supervisors on the progress of development, any successes, changes to the projects requirements, and any challenges that the group has faced recently to get advice on how to tackle them. Emergency meetings were also scheduled when urgent matters arose within the project, such as the loss of a group member as explained later in this chapter. Outside of office hours, contact with the project supervisors was maintained using e-mail.
        
        General group meetings were also arranged weekly to discuss project progress, identify any new or remaining tasks for the current stage of the project and assign these tasks to group members, and re-arrange or re-assign tasks to improve the timetabling of the project. Discussion between members about software design and review of new code was performed regularly to preemptively eliminate potential issues. Any research that had been done since the last meeting was also shared between group members to aid the understanding of the code base further in the event of task re-assignment. These meetings consisted of all 3 group members and would typically be performed within the departmental computer labs, where work could be demonstrated on laptops or departmental computers. Some meetings were also used to test prototype software, and as such required all group members to be present to help set up and operate the virtual reality system.
        
    \subsection{Weekly Project Review}
    
        As per the development methodology chose, at each weekly meeting a review of the project's current state was performed with respect to its timetable, as shown in the Gantt chart in figure~\ref{fig:project_timetable}. This review consisted of individual member updates as to the progress of their assigned tasks. By combining the individual progress of each member, the overall state of progress for the project could be ascertained. A list of things that are to be completed was constructed at each meeting, for completion by the next.
    
    \subsection{Internal Milestones}

        With the group meetings and weekly reviews, a number of internal milestones were set to work toward. These milestones represented a state in the project's development that meant the software met a number of the project requirements. A few of these milestones are displayed in the Gantt chart in figure~\ref{fig:project_timetable} as the prototype milestones and the feature freeze. Some of these milestones also served as a way to prevent feature and scope creep of the project by preventing the introduction of new features after a specific point in the project's life cycle.

\section{Collaboration Tools}
    \subsection{GitHub}
    
        To help with resource management for the project's source code and documentation like this report, several git repositories were created and utilised as source control. GitHub is a web-based git repository host and web front end for the git version control system. It provides a means to simplifying the git workflow through a number of easy to use tools and shortcuts, aiding with tasks such as the merging of code changes into from a development branch into a master branch of a project. It was used to great effect by project, allowing individual members to work on the same repository simultaneously without issue of overwriting another member's work. Previous versions of all code was kept through the version control system and could be accessed in the event a bug or feature regression is introduced into the software. Whilst GitHub offers paid-for features, all of the necessary features for this project was available through their free tier of use.
    
    \subsection{ShareLaTeX}
    
        With the use of \LaTeX~for all project documentation outside of source code, ShareLaTeX provided a way for group members to collaborate writing these documents. ShareLaTeX is a web-based service providing an isolated \LaTeX~environment in which a number of collaborators can write and compile latex documents. It gives users a number of utilities such as auto complete, spell checking, and on-the-fly static analysis of the document sources. On top of their in-built version control, ShareLaTeX also allows users to synchronise their project with a GitHub-hosted repository for version control and hosting. ShareLaTeX helped the group create the documentation necessary for project deliverables with its collaborative tools since each member could work on the document and provide feedback to other member's work in real time. ShareLaTeX uses a pricing model in which the free tier limits projects to one collaborator and doesn't provide GitHub sync. However, one member of the group had already paid for a year's subscription to their 'collaborator' plan, allowing up to 6 collaborators per project for their own personal use. This was leveraged to use ShareLaTeX without cost to the project.
    
    \subsection{Rally}    
        To help in managing the work load and keep track of tasks left to be completed, an online tool called Rally was used~\cite{rally}. The tool is designed to aid in the process of agile development by allowing the users to create user stories, split them into tasks and assign them to different users. The full version of the software requires a purchase, but the Community Edition which was used for this project is a free trial for a limited number of users, which fully met our requirements.
    
        The tool was used to create a list of requirements needed for the project in the form of user stories and collectively distributing those stories between team members. Each individual team member could then divide the user story into a series of tasks that needed to be completed to fulfil the requirement and work on those tasks individually. The team members could also determine time spend on completing a given task, which was helpful in distributing the amount of work between team members in a fair way.
    
    \subsection{Facebook}
    
        With each group member already owning a Facebook account, it was decided that it would be the most convenient method for each group member to communicate with one another. It is possible to share files between people on Facebook, and in the case of small files it is often a more convenient method to do so. Facebook's group messaging system meant that it was easy to broadcast messages to all group members, and as a means to incite group-wide discussion of aspects of the project such as its schedule. 

\section{Risk Management}

    \begin{description}[style=nextline]
            \item[\textbf{VR headset access}]
                The primary concern for development and testing is access to the Virtual Reality hardware. The hardware itself is expensive and therefore the group is unable to source a bespoke set. However, we have mitigated this risk by obtaining permission to access three different sets of the same hardware (the HTC Vive) from the Warick Computing and Game Design societies with access to at least one of these at any given time.
            \item[\textbf{Illness and absence}]
                The risk of team member absence is always great. A key member being unable to attend meetings or development sessions may hamper development significantly. The primary method of mitigating this concern is to reduce the amount of restraint and responsibility given to any one person - while they may have areas of concern all team members should be aware of the state of all constituent parts.
            \item[\textbf{Conflict of Interest}]
                Some team members may have differing views on the driving forces and underlying ideals of the design. The impact of this risk is reduced by giving different people ownership of different constituent parts; the individual areas are arbitrated by the listed person. The areas are compatible enough that a difference between two ideals will not restrict development.
            \item[\textbf{Feature creep}]
                The project itself is intentionally open-ended which leaves room for feature creep, the addition of unnecessary or out-of-scope features which may make the final product more bloated. The implementation of a \emph{feature freeze} (no further feature additions) at the end of Project Week 24 will limit the spread of any feature creep.
        \end{description}

\section{Project Challenges}

    \subsection{A Project Member Leaving}
    
        The greatest challenge that faced this project was the loss of a group member early on in its progression. This group member had expertise in rendering pipelines and the use of the Unity \acrshort{sdk}, making up for a significant part of the technical understanding of the project as originally specified. However, upon their departure it was realised that the specification as it originally stood would not be feasible. To overcome the issues that arose from this member loss, a number of meetings with project supervisors and departmental heads were arranged. After these meetings it was decided that the original specification would be altered by changing the aims of the project to better reflect the capabilities of the remaining group members. These changes involved the reduction of requirements and alteration of project objectives, resulting in the removal of the physics simulation component that was originally proposed. Team roles were re-defined to spread the work that the 4th member originally was planned to take on. The task division of the project was altered only slightly. However, the scheduling of the tasks was altered moreso to allow for group members to research best practices and gain the necessary understanding needed to work with the Unity \acrshort{sdk}.
    
    \subsection{Research}
    
        Another large challenge surrounding the project was researching the tools, techniques, and algorithms that could be used to implement the software produced. A comparison of a few \acrshortpl{sdk} that had \acrshort{vr} capabilities was made in order to decide which one was most appropriate for both the project and its members, these \acrshortpl{sdk} were Unreal Engine and Unity. It was decided that Unity would be used because of its better community support for \acrshort{vr} and the reduced technical investment due to the 4th group member being part of the project at the time of deciding this. Once this project member had left, it then became necessary for every individual within the group to gain a working knowledge of the Unity \acrshort{sdk}.
    
    \subsection{Development}
    
        A working concept for this project was originally created by the 4th group member early on in the project's development, and was used as a base to work upon until their departure. Whilst used as a skeleton, it was decided that its architecture no longer suited the project specification and thus it would be re-written to support future developments within the project. Once this challenge had been overcome, a number of the tasks within the project required the implementation of complex algorithms. Several implementations of these algorithms were found in other languages, and where possible they were used as a basis for implementation in this project. Where this did not occur, the algorithms had to be written from scratch in C\# where they also had to be tested thoroughly.

\section{Legal, Social, Ethical, and Professional Issues}

    \subsection{Legal Issues}
    \label{sec:legal}
        All the Unity assets available through the Unity Asset Store are distributed under the Asset Store Terms of Service and EULA~\cite{unity:terms}. Those terms allow the assets obtained through the store to be used and distributed as integrated part of the interactive media. Library licenses may also change over time, and this will change in the project's code base if dependencies are updated without checking for this. License incompatibilities may become an issue from this.
        
    \subsection{Social Issues}
        While giving the system to the users for testing we bore in mind that there are recorded cases of VR technologies having negative effects including nausea and headaches; a signed waiver would be required of any members of the public to ensure they are aware of these potential problems. As negative effects usually only present when the technology is used for protracted periods (more than 30 minutes) \cite{occulus_health_and_safety}, we were sure to restrict testing sessions to no more than approximately 10 minutes, in accordance with the Healthy and Safety guidelines for the Oculus Rift.
    
    \subsection{Ethical Issues}
        Quite recently Facebook announced their new tool that is designed around virtual reality called Facebook Spaces~\cite{facebook2017}. The tool is as extension of the social network and allows people to meet up virtually and interact with each other using their virtual reality avatars. As the article mentions, the tool has received a lot of negative criticism for promoting antisocial behaviour by discouraging people from communicating in real life. This may be a relevant feedback to us, since similar opinions may be said about a tool that is designed to use in a professional environment. Using our system makes it more difficult for multiple users to cooperate on one project, unless they are all wearing their own head sets. This can also be viewed as a tool that encourages people to interact virtually instead of doing so in real life.
        
    \subsection{Professional Issues}
        Given that the project is being released under an open source license, it is possible that it could be used to plan out acts that are illegal or otherwise unethical. Whilst the license included with the project absolves the creators of issues with the software, the problems that arise from use of software for illegal or unethical means is one that cannot be dealt with easily. If the responsibility lies with the creators to limit the functionality of the software being created, there are major concerns for the free speech that can be expressed by the developers. An example of a possible unethical use of the software would be the use of the software to plan out a break-in by police officers. A similar situation that is also illegal is the planning of an escape from a bank robbery by the robbers.