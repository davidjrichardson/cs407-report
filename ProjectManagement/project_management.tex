When undertaking any project, several challenges may come to light and subsequently must be rigorously accounted for with appropriate documentation during the planning, design, and implementation phases of the project. Some of these challenges could be the organisation of the team members, the selection of what tools that team members shall use, or the software development methodology itself. The purpose of this chapter is to discuss areas of project management such as these, and how the group and its members overcame challenges like them throughout the project's life cycle.

\section{Project Methodology}

    \subsection{Project Summary}        
    
        To gain a clear understanding of any software development project, a few key tasks must be first complete before any development work can be done. The project aims and resources to achieve them must be clearly defined, and the stakeholders in the project should be identified. Once these steps have been taken, it's also crucial to justify the project to each group member as to ensure that they are individually happy with why and how the project will move forward. For this project this has already been discussed in detail within chapter~\ref{chp:specification}, though this section will contain a summary of this specification from the perspective of project management.
        
        The original project idea was proposed and accepted with intrigue and excitement from all group members after an initial enquiry into its feasibility, given that the hardware to test virtual reality software is not common place. Once this concern was addressed with the Computing Society at the university giving the group access to the HTC Vive \acrshort{vr} system. The project gave the group members an opportunity both to work with a novel \acrshort{hci} technology, and to work on the cutting edge of the research into possible industrial uses of \acrshort{vr}. Using the HTC Vive, the group set out with the objective of creating a tool that can be used to aid the architectural design and visualisation process, allowing the user to create and manipulate 3 dimensional spaces in a virtual reality.
        
        With a large project scope, it was decided to develop the software as a proof-of-concept with a minimal list of tools that provided all of the functionality necessary to meet the project objectives and aims. With the project's approval by its supervisors, the following work consisted of determining a use case to work toward, researching \acrshortpl{sdk} and best practises for developing a virtual reality application, and finalising the list of basic tools to be implemented in the project software. Once these tasks were completed, a list of common household items to be included as assets was drawn up. As described in section~\ref{sec:progress_management}, the progress of the project's stages was monitored, with the tasks being scheduled between members both evenly and to suit each member's individual strengths where possible. The project was tested frequently to also monitor potential software problems and functionality regressions as time went on. In the final stages of the project, the departmental project deliverables were consolidated and brought to a conclusion to ensure that it is delivered safely whilst also addressing the project's previously established objectives.

    \subsection{Software Development Methodology}
    
        As with any software development project, it is necessary to consider enacting an appropriate development methodology. If not, the lack of effective practices can lead to unpredictable functionality, poor code quality and documentation, and repeated error. Software development methodologies range from the rigid-structure, document centric waterfall approach to more agile methods that prioritise adaptability to requirement changes and code quality.
    
        % We are a small group w/ timetables that are changing frequently -> need something flexible
        % A small number of members -> high likelihood of group work being delayed -> documentation needed to allow for members to pick up
        % Project requirements might change through user or supervisor feedback -> methodology needs ability to change requirements
        % We used agile 

\section{Team Structure}
    
    % State who did what in the team, describing briefly what they did

\section{Time \& Task Management}

\section{Progress Management}
\label{sec:progress_management}

    \section{Meetings}
    
    \section{Internal Milestones}

% The bs section about rally, if it's too much we can always get rid of it
\section{Collaboration Tools}
    % TODO: Add sections for all tools here
    \subsection{GitHub}
    
    \subsection{ShareLaTeX}
    
    \subsection{Facebook}
    
    \subsection{Rally}
        To help in managing the work load and keep track of tasks left to be completed, an on-line tool called Rally was used~\cite{rally}. The tool is designed to aid in the process of agile development by allowing the users to create user stories, split them into tasks and assign them to different users. The full version of the software requires a purchase, but the Community Edition which was used for this project is a free trial for a limited number of users, which fully met our requirements.
    
        The tool was used to create a list of requirements needed for the project in the form of user stories and collectively distributing those stories between team members. Each individual team member could then divide the user story into a series of tasks that needed to be completed to fulfil the requirement and work on those tasks individually. The team members could also determine time spend on completing a given task, which was helpful in distributing the amount of work between team members in a fair way.

\section{Project Challenges}

    \subsection{A Project Member Leaving}
    
    \subsection{Research}
    
    \subsection{Development}

\section{Risk Management}

    % TODO: Paste the risk matrix for the project in here

\section{Legal, Social, Ethical, and Professional Issues}
    \subsection{Legal Issues}
    \label{sec:legal}
        All the Unity assets available through the Unity Asset Store are distributed under the Asset Store Terms of Service and EULA~\cite{unity:terms}. Those terms allow the assets obtained through the store to be used and distributed as integrated part of the interactive media.
    
    \subsection{Ethical Issues}
        Quite recently Facebook announced their new tool that is designed around virtual reality called Facebook Spaces~\cite{facebook2017}. The tool is as extension of the social network and allows people to meet up virtually and interact with each other using their virtual reality avatars. As the article mentions, the tool has received a lot of negative criticisms for promoting antisocial behaviour by discouraging people from communicating in real life. This may be a relevant feedback to us, since similar opinions may be said about a tool that is designed to use in a professional environment. Using our system makes it more difficult for multiple users to cooperate on one project, unless they are all wearing their own head sets. This can also be viewed as a tool that encourages people to interact virtually instead of doing so in real life.