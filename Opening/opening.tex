\section{Key Words}
    Virtual Reality, Human-computer Interaction, Architecture Visualisation.

\section{Acknowledgements}

    We would like to thank a number of people that have helped us through the extensive research and development that was performed as part of this project. Our project supervisors Arshad Jhumka and Hongkai Wen, and Dr. Matthew Leeke all provided counsel when a key group member left the project. The Warwick Manufacturing Group's VR research team let us use their HTC Vive to initially prototype a tool as part of our feasibility study. Similarly, the Computing and Game Design societies at the university allowed us to utilise their virtual reality hardware for development and testing purposes.
    
\section{Introduction}

    The concept of \acrfull{vr} has recently come into the spotlight of computer hardware manufacturers with an increasing variety of high-quality \acrshort{vr} systems such as the HTC Vive or Oculus Rift. Whilst these systems are aimed toward the computer games market, there is a wide range of commercial and industrial uses for them as well. This project aims to create a \acrshort{vr}-based proof-of-concept tool for the area of architectural visualisation. Whilst the tool is proof-of-concept, it is hoped that concepts implemented in it are translated into production software solutions.
    
    This report will give a comprehensive analysis of the project undertaken by our group. The report will start with a background summary of the components relating to its field as well as an outline of current research, development, and production being undertaken. The project's specifications will then be elicited and justified. Moving forward, current literature will be reviewed to determine this project's place within it. The report will then detail the design, implementation, and testing of our software tool, including the project's management and any considerations made. Finally, the project's outcome will be evaluated and a conclusion given that reflects upon the project's success, giving insight into future works.